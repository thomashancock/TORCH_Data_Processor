\subsection*{Introduction}

This program reads raw T\+O\+R\+CH data and processes it into a R\+O\+OT file.

Test data files are provided to aid in development of the program.

For an overview of the program, see \href{https://indico.cern.ch/event/731827/contributions/3026751/attachments/1660291/2659581/Multiboard_Data_Processor.pdf}{\tt slides from the Testbeam Meeting on 1st June 18}.

The {\ttfamily Documentation} directory contains full deoxygen documentation. It can be viewed through {\ttfamily open Documentation/html/index.\+html}.

Please report any issues to Thomas Hancock (\href{mailto:thomas.hancock@physics.ox.ac.uk}{\tt thomas.\+hancock@physics.\+ox.\+ac.\+uk}).

\subsection*{Requirements to build the processor}

The Multiboard Data \hyperlink{class_processor}{Processor} (M\+DP) requires\+:


\begin{DoxyItemize}
\item A working installation of C\+Make
\item A C++14 compliant compiler (e.\+g. gcc 4.\+9 or greater)
\item A working installation of R\+O\+OT 6
\end{DoxyItemize}

Note\+: The program has been tested with root 6.\+08.\+04, but theoretically any version of R\+O\+OT 6 should work (if you have issues related to this, contact Thomas Hancock).

\subsection*{Compiling the \hyperlink{class_processor}{Processor}}

The Multiboard Data \hyperlink{class_processor}{Processor} (M\+DP) is built using C\+Make.

To build the M\+DP, do\+: 
\begin{DoxyCode}
mkdir build
cd build
cmake .. -DCMAKE\_INSTALL\_PREFIX=$(install\_path) -DCMAKE\_BUILD\_TYPE=Release
make
make install
\end{DoxyCode}


The M\+DP will be built in the {\ttfamily bin} directory of the set {\ttfamily }.

If developing the processor, the program can be build in debug mode by changing the C\+Make Build Type\+: 
\begin{DoxyCode}
-DCMAKE\_BUILD\_TYPE=Debug
\end{DoxyCode}


\subsection*{Running the Program}

To run the M\+DP, simply call the program\+: 
\begin{DoxyCode}
./$(install\_path)/bin/TORCH\_Data\_Processor
\end{DoxyCode}
 followed by the files you wish to process (e.\+g. {\ttfamily data/$\ast$.txt})

Data files must contain the string {\ttfamily Device\+\_\+\+\_\+\+\_\+} in their name. This allows the M\+DP to properly synchronise data across multiple readouts. The string must be enclosed by underscores, but can contain any words on either side.

The program is configured via a xml config file. This file is specified with the {\ttfamily -\/con} command line option. If not set, the program will search for {\ttfamily Config.\+xml}. An example configuration file is provided for you to modify ({\ttfamily Example\+\_\+\+Config.\+xml}).

The output file name can be specified {\ttfamily -\/out} command line argument. If not set, the output will be called {\ttfamily Output.\+root}.

\subsection*{Running a quick test}

A script is provided which runs a quick test of the M\+DP in {\ttfamily Serial} mode.

To run the test, do\+: 
\begin{DoxyCode}
. run\_test.sh
\end{DoxyCode}


This runs the M\+DP on the data contained in {\ttfamily ./data/\+Test\+\_\+\+Data} using the config specified by {\ttfamily Example\+\_\+\+Config.\+xml}.

The test data is deliberately not perfect. The resulting output should show several \char`\"{}\+Dumping events due to buffer bloat\char`\"{} warnings, and the Errors Summary should contain a large number of errors (of types \char`\"{}\+Bad Packet Dumped\char`\"{}, \char`\"{}\+Dumped incomplete packet\char`\"{}, \char`\"{}\+Word found out of sequence\char`\"{}).

\subsection*{Pulling new changes}

New versions of the M\+DP can be acquired via git. When pulling or fetching a new version of the processor, please ensure you do 
\begin{DoxyCode}
git fetch --tags
\end{DoxyCode}
 to make your local tags up to date with the Git\+Lab tags. The version tag is stored in the output root file to ensure replicability of results, and having an incorrect local tag destroys this feature.

In order to properly record the new tag, the M\+DP must be rebuilt from scratch. Please ensure you do 
\begin{DoxyCode}
make clean
\end{DoxyCode}
 in the build directory before rebuilding the program to ensure the tag is correctly propagated.

\subsection*{To-\/do}

This section gives a list of changes/features which are yet to be implemented, but are requested. The requester\textquotesingle{}s initials should be put in brackets after the item.


\begin{DoxyItemize}
\item Add possibility of logging errors in \hyperlink{class_error_spy}{Error\+Spy} which don\textquotesingle{}t required a Board ID and/or T\+DC ID (T\+HH)
\item Refactor \hyperlink{class_edge}{Edge} Sorting to be easily modifiable (T\+HH)
\item Add \char`\"{}\+No Errors\char`\"{} to \hyperlink{class_error_spy}{Error\+Spy} summary if no errors are detected (T\+HH)
\item Parallel Mode (T\+HH)
\end{DoxyItemize}

\subsection*{Known Bugs}

This section gives a list of known bugs which require attention. The reporter\textquotesingle{}s initials should be put in brackets after the item.


\begin{DoxyItemize}
\item If block/file is dumped, ensure current \hyperlink{class_word_bundle}{Word\+Bundle} should be cleared (T\+HH)
\item \char`\"{}\+Invalid Datatype\char`\"{} Error has T\+DC ID = 100 for all occurrences (T\+HH) 
\end{DoxyItemize}