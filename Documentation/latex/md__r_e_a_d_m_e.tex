\subsection*{Quick Reference}


\begin{DoxyEnumerate}
\item \href{#Introduction}{\tt Introduction}
\item \href{#Requirements}{\tt Requirements}
\item \href{#Compiling}{\tt Compiling the Processor}
\item \href{#Running}{\tt Running the Processor}
\begin{DoxyEnumerate}
\item \href{#RunQuickTest}{\tt Running a quick test}
\item \href{#RunLongRuns}{\tt Running over long data runs}
\end{DoxyEnumerate}
\item \href{#Pulling}{\tt Pulling New Changes}
\item \href{#Macros}{\tt Macros}
\item \href{#ToDo}{\tt To-\/do}
\item \href{#KnownBugs}{\tt Known Bugs}
\end{DoxyEnumerate}

\subsection*{Introduction \label{_Introduction}%
}

This program reads raw T\+O\+R\+CH data and processes it into a R\+O\+OT file.

For an overview, see \href{https://indico.cern.ch/event/731827/contributions/3026751/attachments/1660291/2659581/Multiboard_Data_Processor.pdf}{\tt slides from the Testbeam Meeting on 1st June 18}.

The {\ttfamily Documentation} directory contains full deoxygen documentation. It can be viewed through {\ttfamily open Documentation/html/index.\+html}.

Please report any issues to Thomas Hancock (\href{mailto:thomas.hancock@physics.ox.ac.uk}{\tt thomas.\+hancock@physics.\+ox.\+ac.\+uk}).

\subsection*{Requirements \label{_Requirements}%
}

The Multiboard Data \hyperlink{class_processor}{Processor} (M\+DP) requires\+:


\begin{DoxyItemize}
\item A working installation of C\+Make
\item A C++14 compliant compiler (e.\+g. gcc 4.\+9 or greater)
\item A working installation of R\+O\+OT 6
\end{DoxyItemize}

Note\+: The program has been tested with root 6.\+08.\+04, but any version of R\+O\+OT 6 should work (if you have issues related to this, contact Thomas Hancock).

\subsection*{Compiling the \hyperlink{class_processor}{Processor} \label{_Compiling}%
}

The Multiboard Data \hyperlink{class_processor}{Processor} (M\+DP) is built using C\+Make.

To build the M\+DP, do\+: 
\begin{DoxyCode}
mkdir build
cd build
cmake .. -DCMAKE\_INSTALL\_PREFIX=$(install\_path) -DCMAKE\_BUILD\_TYPE=Release
make
make install
\end{DoxyCode}


The M\+DP will be built in the {\ttfamily bin} directory of the set {\ttfamily }.

If developing the processor, the program can be build in debug mode by changing the C\+Make Build Type\+: 
\begin{DoxyCode}
-DCMAKE\_BUILD\_TYPE=Debug
\end{DoxyCode}


\subsection*{Running the \hyperlink{class_processor}{Processor} \label{_Running}%
}

To run the M\+DP, simply call the program\+: 
\begin{DoxyCode}
./$(install\_path)/bin/TORCH\_Data\_Processor
\end{DoxyCode}
 followed by the files you wish to process (e.\+g. {\ttfamily data/$\ast$.txt})

Data files must contain the string {\ttfamily \+\_\+\+Chain\+\_\+\+\_\+\+Device\+\_\+\+\_\+\+\_\+} in their name. This allows the M\+DP to properly synchronise data across multiple readouts. The string can be surrounded by words on either side, but any words which follow must be separated from the string by an underscore.

The program is configured via a xml config file. This file is specified with the {\ttfamily -\/con} command line option. If not set, the program will search for {\ttfamily Config.\+xml}. An example configuration file is provided for you to modify ({\ttfamily Example\+\_\+\+Config.\+xml}).

The output file name can be specified {\ttfamily -\/out} command line argument. If not set, the output will be called {\ttfamily Output.\+root}.

\subsubsection*{Running a quick test \label{_RunQuickTest}%
}

A script is provided which runs a quick test of the M\+DP in {\ttfamily Serial} mode.

To run the test, do\+: 
\begin{DoxyCode}
. run\_test.sh
\end{DoxyCode}


This runs the M\+DP on the data contained in {\ttfamily ./data/\+Test\+\_\+\+Data} using the config specified by {\ttfamily Example\+\_\+\+Config.\+xml}.

The test data is deliberately not perfect. The resulting output should show the following error summary\+: 
\begin{DoxyCode}
=== Errors Summary ===
Error: Invalid Datatype 6
        ChainID: 0, DeviceID: 0 (x 11)
Error: Invalid Datatype 7
        ChainID: 0, DeviceID: 0 (x 11)
Error: Invalid Datatype 14
        ChainID: 0, DeviceID: 0 (x 45)
        ChainID: 1, DeviceID: 0 (x 177)
\end{DoxyCode}


\subsubsection*{Running over long data runs \label{_RunLongRuns}%
}

Attempting to run over large numbers of files (more than $\sim$1000) may result in the following error\+: 
\begin{DoxyCode}
-bash: ./Install/bin/DataProcessor: Argument list too long
\end{DoxyCode}


In this case, an alternative method for passing data files to the M\+DP is provided. In the {\ttfamily scripts} directory is a python script which, when passed one or more directories, will produce a newline separated list of files. E.\+g. 
\begin{DoxyCode}
python scripts/MakeFileList.py data/Test\_Data/TimeRef\_01\_Device\_?
\end{DoxyCode}


The output is stored in {\ttfamily filelist.\+lst}\+: 
\begin{DoxyCode}
data/Test\_Data/TimeRef\_01\_Device\_0/TimeRef\_01\_Device\_0\_000000\_20180523110722.txt
data/Test\_Data/TimeRef\_01\_Device\_0/TimeRef\_01\_Device\_0\_000001\_20180523110738.txt
data/Test\_Data/TimeRef\_01\_Device\_0/TimeRef\_01\_Device\_0\_000002\_20180523110753.txt
data/Test\_Data/TimeRef\_01\_Device\_0/TimeRef\_01\_Device\_0\_000003\_20180523110809.txt
data/Test\_Data/TimeRef\_01\_Device\_0/TimeRef\_01\_Device\_0\_000004\_20180523110823.txt
...
\end{DoxyCode}


File lists can be passed to the M\+DP using the {\ttfamily -\/list} option. E.\+g. 
\begin{DoxyCode}
./Install/bin/DataProcessor -con Config.xml -list filelist.lst
\end{DoxyCode}


The M\+DP will run over all the files given in {\ttfamily filelist.\+lst} as if they were passed on the command line.

\subsection*{Pulling New Changes \label{_Pulling}%
}

New versions of the M\+DP can be acquired via git. When pulling or fetching a new version of the processor, please ensure you do 
\begin{DoxyCode}
git fetch --tags
\end{DoxyCode}
 to make your local tags up to date with the Git\+Lab tags. The version tag is stored in the output root file to ensure replicability of results, and having an incorrect local tag destroys this feature.

In order to properly record the new tag, the M\+DP must be rebuilt from scratch. Please ensure you do 
\begin{DoxyCode}
make clean
\end{DoxyCode}
 in the build directory, then follow the steps in the \char`\"{}\+Compiling the Processor\char`\"{} section from the call to {\ttfamily cmake} onwards.

\subsection*{Macros \label{_Macros}%
}

R\+O\+OT macros for quickly processing the M\+DP output can be found in the {\ttfamily macros} directory. Each take the file to run over as input, and can be run on the output of both {\ttfamily Low\+Level} and {\ttfamily Serial} mode.

E.\+g. 
\begin{DoxyCode}
root 'macros/MakeChannelHisto.cxx("Output.root")' -b -q
\end{DoxyCode}


Currently two macros are provided\+:


\begin{DoxyItemize}
\item {\bfseries Make\+Channel\+Histo}
\end{DoxyItemize}

Makes a histogram of the hits on each channel, separation out time reference channels according to the {\ttfamily Std8x64\+Mapping} Channel Mapping.

The output will be called {\ttfamily Channel\+I\+D\+Histogram.\+pdf}.

An additional argument can be provided to set the y axis scale to be logarithmic. In order to do this, add a 1 after the input file. E.\+g. 
\begin{DoxyCode}
root 'macros/MakeChannelHisto.cxx("Output.root",1)' -b -q
\end{DoxyCode}



\begin{DoxyItemize}
\item {\bfseries Make\+Hitmap}
\end{DoxyItemize}

Makes a hitmap of detected hits, assuming the reference channels are laid out according to the {\ttfamily Std8x64\+Mapping} Channel Mapping.

The output will be called {\ttfamily Hitmap.\+pdf}.

Like {\ttfamily Make\+Channel\+Histo}, additional optional arguments can be passed. The first extra argument will set the z axis to a logarithmic scale if set to {\ttfamily 1}, and the second will exclude the time reference channels. E.\+g. 
\begin{DoxyCode}
root 'macros/MakeHitmap.cxx("Output.root",1,1)' -b -q
\end{DoxyCode}
 will create a hitmap with a logarithmic z axis and no time references included.

To manually specify the output file name, pass the desired name as the second argument (i.\+e. after the input file name). The remaining optional arguments follow as previosuly detailed. E.\+g. 
\begin{DoxyCode}
root 'macros/MakeHitmap.cxx("Output.root","CustomOutputName.pdf",1,1)' -b -q
\end{DoxyCode}


\subsection*{To-\/do \label{_ToDo}%
}

This section gives a list of changes/features which are yet to be implemented, but are requested. The requester\textquotesingle{}s initials should be put in brackets after the item.


\begin{DoxyItemize}
\item Refactor \hyperlink{class_edge}{Edge} Sorting to be easily modifiable (T\+HH)
\item Parallel Mode (T\+HH)
\end{DoxyItemize}

\subsection*{Known Bugs \label{_KnownBugs}%
}

This section gives a list of known bugs which require attention. The reporter\textquotesingle{}s initials should be put in brackets after the item.


\begin{DoxyItemize}
\item If block/file is dumped, current \hyperlink{class_word_bundle}{Word\+Bundle} should be cleared (T\+HH) 
\end{DoxyItemize}