\subsection*{Introduction}

This program reads raw T\+O\+R\+CH data and processes it into a R\+O\+OT file.

Test data files are provided to aid in development of the program.

The {\ttfamily Documentation} directory contains an overview of the program, along with full deoxygen documentation.

Please report any issues to Thomas Hancock (\href{mailto:thomas.hancock@physics.ox.ac.uk}{\tt thomas.\+hancock@physics.\+ox.\+ac.\+uk}).

\subsection*{Compiling the Template}

To build the program, simply type\+: 
\begin{DoxyCode}
make
\end{DoxyCode}


If developing the processor, the program can be build in debug mode using\+: 
\begin{DoxyCode}
make debug
\end{DoxyCode}


Note\+: A working installation of R\+O\+OT is required to build the program

\subsection*{Running the Program}

To run the template, simply call the program\+: 
\begin{DoxyCode}
./TORCH\_Data\_Processor
\end{DoxyCode}
 followed by the files you wish to process (e.\+g. {\ttfamily data/$\ast$.txt})

The {\ttfamily -\/con} command line option allows a specific config ({\ttfamily .conf}) file to be loaded.

The {\ttfamily -\/out} command line argument specifies the output file name.

\subsection*{Pulling new changes}

When pulling or fetching a new version of the processor, please ensure you do 
\begin{DoxyCode}
git fetch --tags
\end{DoxyCode}
 to make your local tags up to date with the Git\+Lab tags. The version tag is stored in the output root file to ensure replicability of results, and having an incorrect local tag destroys this feature.

\subsection*{To-\/do}

This section gives a list of changes/features which are yet to be implemented, but are requested. The requester\textquotesingle{}s initials should be put in brackets after the item


\begin{DoxyItemize}
\item Prevent header/trailer word overwriting in \hyperlink{class_packet}{Packet} class (T\+HH)
\item Add global monitoring for caught issues and inconsistencies (T\+HH) 
\end{DoxyItemize}